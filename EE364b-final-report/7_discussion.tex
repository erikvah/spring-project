\subsection{Performance}
As shown in section~\ref{sec:result}, Kruskal's method for stress minimization is fast and reliable. However, for the problem sizes investigated in this project, the speed of all implemented algorithms was fast enough that it does not matter in practice. Due to the limited speed and memory of the laptop that all tests and simulations were carried out on, larger problem sizes were not feasible to run the Riemannian elevator on because of the SDP optimization step. While this is the case, it should still be noted that Kruskal's method is fast and stable for all problem sizes investigated, and seems likely to remain so should much larger problems be investigated. 

While the Riemannian is more reliable than a random initial guess, the difference is smaller than expected. When the comparatively long runtime is factored in, it is not unlikely that a better success rate in a shorter time might be achieved by running multiple trials of random guesses, especially if they are ran in parallel. Part of the performance gap might partially be explained by implementation details, as the gradient descent steps were implemented in an optimized fashion specifically for this problem, while the SDP and manifold optimization steps used the general-purpose solver libraries CVXPY~\cite{cvxpy1,cvxpy2} and Pymanopt~\cite{pymanopt}, respectively. It is likely that the computational limit on problem size would be possible to circumvent by using another solver algorithm or library. 