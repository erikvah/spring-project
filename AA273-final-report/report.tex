\documentclass[conference]{IEEEtran}

\usepackage{amsmath,amssymb,amsfonts} % Math
\usepackage{bm} % Bold greek letters
% \usepackage{algorithmic}
\usepackage{graphicx} % Plots
\usepackage[style=ieee]{biblatex} % Bibliography management
% \usepackage{textcomp}
\usepackage[super]{nth} % Support for nth counting, aka 1st, 2nd, ...
\usepackage{xcolor}
\usepackage{algorithm, algpseudocode} % Algorithm block
\usepackage{setspace} % Spacing in algorithm block
\usepackage{tabularx} % Good tables
\usepackage{placeins} % FloatBarrier 
\usepackage{subcaption} % Subfigures
\usepackage{svg}
\usepackage{hyperref}

\addbibresource{..\\sources.bib}

\graphicspath{{.\\figures}}

\newcommand*{\norm}[1]{\left|\left|#1\right|\right|}

\begin{document}

\title{Range-Based Localization and Tracking Using Hybrid Optimization}

\author{\IEEEauthorblockN{Erik Helmer}
\IEEEauthorblockA{\textit{Department of Electrical Engineering} \\
\textit{Stanford University}\\
Palo Alto, USA \\
erik.helmer@stanford.edu}
}

\maketitle

\begin{abstract}
This project intends to investigate methods for solving the classic problem of range-based localization. It uses novel robust algorithms to find an optimal starting point, then uses faster methods for non-convex optimization in later time steps given the good initial guess. The goal is to estimate the trajectories of a group of robots over time. 

\end{abstract}

\begin{IEEEkeywords}
wireless sensor networks, localization, optimization, estimation
\end{IEEEkeywords}

\section{Introduction}
\label{sec:intro}
In a variety of applications, it is necessary to localize a group of sensors or robots. In open-air environments, this can be done with GPS. In enclosed where GPS signal is not available, it is possible to do this using a variety of SLAM techniques such as bundle adjustment or pose graph optimization (PGO). These can and have been extended to optimize over a group of robots, simultaneously determining their positions~\cite{SLAM_distributed}. However, these are dependent on a complex signal processing pipeline to extract landmark or odometry measurements to optimize over. This project intends to investigate a different approach. 

The subject of this project is an implementation and extension of a novel optimization technique for distance-based relative localization. This problem is described in figure~\ref{fig:problem_desc}.
\begin{figure}[ht]
    \centering
    \includegraphics[width=\linewidth,trim=31mm 48mm 106mm 15mm, clip]{problem.pdf}
    \caption{Basic problem setup. In the figure, we have $N=4$ robots where $\mathbf{x}^{(i)}_t \in \mathbf{R}^n$ are the poses of the robots and $y^{(i,j)}_t \in \mathbf{R}$ are the distance measurements between them.}
    \label{fig:problem_desc}
\end{figure}

SLAM  methods usually rely on measurements of external fixed points in the scene to extract measurements. This method would instead rely on the robots measuring only distance to each other. 

Some assumptions are made of the problem. It is assumed that while the distance measurements may be noisy, they are measurements of another robot in the group and not a false measurement of the environment. We will further assume additive Gaussian noise on the measurements. This project will not investigate the distributed case, and will instead assume centralized knowledge of all robot measurements. While none of the algorithms used in this project require it to be so, we have nonetheless chosen to only investigate the two-dimensional case of the problem for simplicity. 

\section{Related Work}
\label{sec:related-work}
The problem formulated in section~\ref{sec:intro} is usually in the literature reformulated as a graph problem, see figure~\ref{fig:problem-graph}. The measurements are interpreted as the edge weights and the positions are interpreted as vertex positions. 
% \begin{figure}[ht]
%     \centering
%     \includegraphics[width=\linewidth,trim=31mm 48mm 106mm 15mm, clip]{graph.pdf}
%     \caption{Problem reformulation. In the figure, we have a graph $\mathcal{G}=(\mathcal{V}, \mathcal{E})$ where the edges $(i, j) \in \mathcal{E}$ have weights $y_{i,j}$ and the vertices $k \in \mathcal{V}$ have positions $\bar{x}_k$.}
%     \label{fig:problem-graph}
% \end{figure}

In practice, it is rarely the case that the correspondences between robots are known without sophisticated identification algorithms using computer vision or other computationally complex methods. This is known as the anonymous localization problem, and it has been investigated by Franchi, Oriolo and Stegagno~\cite{anonymous_loc_1,anonymous_loc_2,anonymous_loc_3}. They developed a few algorithms to account for the missing correspondences by using statistical methods to simultaneously estimate multiple possible relative poses to determine the most likely correspondences. However, this markedly increases the complexity of the problem and will not be the focus of this project.

It will henceforth generally be assumed that while the measurements $y_{i,j}$ are not complete, i.e. all robots do not have measurements of every other robot, the correspondences are known. That is to say, for all measurements $y$, it is known which distance between two robots this is a measurement of. 

This problem is not new. One early approach to solve it was presented by Kruskal in \cite{Kruskal1964}. He proposed a measurement of good fit, or loss, named the stress function: 
\begin{align}
    \label{eq:kruskal-stress}
    S(x^{(1)}, ..., x^{(N)}) = \sum_{i,j} \left(
        \norm{x^{(i)} - x^{(j)}}_2 - y^{(i, j)}
    \right)^2
\end{align}
More accurately, Kruskal proposed a normalized version of equation~\ref{eq:kruskal-stress}, but for the purposes of this project the unnormalized version is sufficient. This function is non-convex, so iterative optimization algorithms such as gradient descent or ADMM generally converge to local minima, which necessitates a good initial guess. This algorithm will hereafter be called stress minimization in this paper.

In a companion paper, Kruskal provided a description of how to solve this problem using gradient descent~\cite{kruskal1964implementation}. Of interest to this project is the gradient that Kruskal determined. It should be noted that the stress function is non-differentiable and non-convex, so while Kruskal calls it a gradient, it is not necessarily so. Nonetheless, the stress minimization algorithm presented by Kruskal show empirically good results, so this this paper will continue with the slight abuse of notation. In addition to how do calculate the gradient of the stress function, the paper also presents a way to determine the step sizes in gradient descent. For a complete description of the algorithm, see~\cite{kruskal1964implementation}.

An alternative algorithm was proposed in~\cite{MDS_proposal} by De Leeuw. THe algorithm is called Scaling by MAjorizing a COmplicated Function (SMACOF), and it iteratively minimizes an upper bound of the stress function. Further investigated by De Leeuw in~\cite{SMACOF_convergence}, it has been proven to converge to a local minima of the stress function. However, this also requires a good initial guess to converge to a good optimum. 

These methods, while prone to local minima, can still be useful due to their simplicity when given a sufficiently good initial guess. 

In the field of sensor network localization, meaning many agents are deployed in an environment without knowledge of their positions, this problem has also been investigated. The main distinction with the robot localization problem is that authors generally assume that at least some positions known as anchors are known~\cite{WSN_collaborative,WSN_localization_techniques,optimization_WSN,WSN_stochastic}. 

% Authors in these fields present additional methods of solving this problem, with different advantages and drawbacks. One technique is Particle Swarm Optimization (PSO), where a collection of particles explore the n-dimensional space to find optima~\cite{WSN_particles}. Building on this, Zhou and Chen gave a stochastic approach in~\cite{WSN_stochastic}, which finds the global optimum with high probability. 

Recently, there has been some advancements in solving the distance-based localization problem. An algorithm developed by Halsted and Schwager dubbed the Riemannian Elevator has been proposed which has better guarantees than either stress minimization or SMACOF. A full description of the algorithm is outside the scope of this paper, but in essence, instead of optimizing over the positions directly, it instead optimizes over an expanded $r$-dimensional Riemannian manifold of unit vectors for $r >= 2$. The solution is then projected into $2$ dimensions. It also provides a non-trivial lower bound on the provided solution. For a more thorough description, see~\cite{R_elevator}. 

\section{Problem Statement}
\label{sec:prob-statement}
This project will combine some earlier approaches described in section~\ref{sec:related-work} to investigate whether this can improve estimation performance. The problem to be investigated, as well as the notation to be used, is described below.

\subsection{State-Space Model}
As described in section~\ref{sec:intro}, this projects seeks to investigate the problem of $n$ robots moving around in an unknown open space. The dynamics of the robots will be assumed to be \nth{1} order differential drive robot with positions $x^{(i)}_t$ and orientations $\theta^{(i)}$ which are  controlled by speed $v$ and angular velocity $\omega$. This means that for each robot $i$ at time step $t$ we have 
\begin{align}
    \state = \begin{bmatrix}
        \x \\ \theta
    \end{bmatrix}
    \in \R{3} \quad \text{with} \quad \x \in \R{2}, \thet \in \R{}
\end{align}
A combination of position and angle like this is called pose in SLAM literature. This gives the total state
\begin{align}
    \mathbf{x}_t = \begin{bmatrix}
        \state[1] \\ \vdots \\ \state[n] 
    \end{bmatrix}
    \in \R{3n}
\end{align}
It is controlled by the control signal 
\begin{align}
    \mathbf{u}_t^{(i)} = \begin{bmatrix}
        v_t^{(i)} \\ \omega _t^{(i)}
    \end{bmatrix}
\end{align}
with the total control vector 
\begin{align}
    \mathbf{u}_t = \begin{bmatrix}
        \mathbf{u}_t^{(1)} \\ \vdots \\ \mathbf{u}_t^{(N)}
    \end{bmatrix}
\end{align}
We also have the update equation
\begin{align}
    f(\mathbf{x}_t^{(i)}, \mathbf{u}_t^{(i)}) &= \mathbf{x}_t^{(i)} + \Delta t
    \begin{bmatrix} 
        v_t^{(i)} \cos \theta_t^{(i)} \\ v_t^{(i)} \sin \theta_t^{(i)} \\ \omega_t^{(i)}
    \end{bmatrix}
\end{align}

\subsection{Graph Optimization} \label{sec:graphs}
% $\exists (i, j) \notin \mathcal{E}$
The positions of the robots as well as the distance measurements are modeled as a directed graph $\mathcal{G} = (\mathcal{V}, \mathcal{E})$, with $n$ vertices $i \in \mathcal{V}$ and $m$ edges $(i, j) \in \mathcal{E}$. An example configuration is shown in figure~\ref{fig:problem-graph}.
\begin{figure}[ht]
    \centering
    \includegraphics[width=\linewidth,trim=31mm 48mm 106mm 15mm, clip]{graph.pdf}
    \caption{Problem reformulation. In the figure, we have a graph $\mathcal{G}=(\mathcal{V}, \mathcal{E})$ where the directed edges $(i, j) \in \mathcal{E}$ have weights $l_{i,j}$ and the vertices $k \in \mathcal{V}$ have positions $x^{(k)}_t$.}
    \label{fig:problem-graph} 
\end{figure}
As shown in the figure, the graph need not be fully connected, and up to two edges can connect any two vertices, with this representing that two robots have distance measurements of each other. Additionally, the robot distance measurements are assumed to be taken with Gaussian white noise: 
\begin{align}
    y_t^{(i, j)} = l_t^{(i,j)} + U_t^{(i, j)}, \quad U_t^{(i, j)} \sim \text{GWN}(0, \psi_t^{(i, j)})
\end{align}

We use the notation defined in table~\ref{tab:notation} below to describe the graph and the data collected from it.
\FloatBarrier
\begin{table}[ht]
    \centering
    \caption{Additional definitions}
    \label{tab:notation}
    \begin{tabularx}{\linewidth}{lX}
        $\mathbf{J}_t \in \R{m \times 2}$ & 
        \textit{Connectivity matrix}.  Each row $(i, j)\in\mathcal{E}$ represents a directed edge from $i$ to $j$ of the graph. \\
        $\mathbf{y}_t \in \R{m}$ & 
        \textit{Measurement vector}. Each row $k$ is the distance measurement between the two nodes at row $k$ in $\mathbf{J}_t$. \\
        $\bm{\psi}_t \in \R{m}$ &
        \textit{Noise variance vector}. Each row $k$ is the variance of the noise measurement at row $k$ in $\mathbf{y}_t$. 
    \end{tabularx}
\end{table}


\section{Pipeline}
\label{sec:pipeline}
The pipeline for estimating the state over time uses two different optimization techniques as well as an extended Kalman filter. To initialize the estimation process, the Riemannian elevator algorithm is run once to generate estimations of the positions of the robots based only on distance measurements. These estimations are in practice never optimal points of the stress function, so they are refined by running gradient descent with step sizes according to Kruskal's procedure. 

The result of this is used as the prior $\mathbf{x}_0$, where the angles $\theta_0^{(i)}$ are initialized randomly. An extended Kalman filter (EKF) is then initialized with these mean priors, and high angle uncertainty, which completes the initialization step. The Kalman filter and its extensions try to estimate the state of a system with alternating prediction and update steps. They provide an estimate and the uncertainty of that estimate of a state (mean and covariance in literature), given measurements of that state. A familiarity with Bayesian filtering in general, and the EKF in particular, will henceforth be assumed. 

While tracking the robots, a combination of the EKF and Kruskal gradient descent is used. In accordance with~\cite{R_elevator}, this paper uses the weighted stress function in the gradient descent, which trivially changes the gradients defined in section~\ref{sec:related-work} by adding weights to each element in the sum. The predicted robot poses $\hat{\mathbf{x}}_{t+1 \mid t}$ based on the current pose estimations $\hat{\mathbf{x}}_{t \mid t}$ and the control signals $\mathbf{u}_t$ is determined by the EKF prediction step. After the robots have then moved and taken new distance measurements at the next time, the position prediction is used as the initial guess for Kruskal's gradient descent algorithm. The optimal positions $\mathbf{z}_t$ are then fed to the EKF as the measurements, after which the process repeats. 

The functions used in the EKF are shown below, where the index $(i)$ has been dropped for readability. These functions describe the change of the state of the robots from one time step and the next, and the measurement of that state at a given time step.
% To estimate the state $\totstate$ over time we will use a extended Kalman filter (EKF). However, it will not use the distances between the robots as measurement variable. Instead, it will use the estimated positions that the optimizer algorithms determine. All robots operate independently, but we will use a single filter to estimate all of their states. With this said, update equations for a single robot $i$ are defined, below where the index $(i)$ is dropped for readability:
\begin{align}
    \mathbf{x}_{t+1} &= \f(\mathbf{x}_t, \totu) + \W \\
    \totmeas[t+1] &= \g(\totstate) + \V
\end{align}
where
\begin{align}
    g(\totstate) &= \begin{bmatrix}
        1 & 0 & 0 \\
        0 & 1 & 0
    \end{bmatrix} \totstate \\
    \W &\sim \text{GWN} (0, \Q) \\
    \V &\sim \text{GWN} (0, \R) 
\end{align}
and where $\W$ is uncorrelated with $\V$, as well as both being uncorrelated with noise in different robots. By linearizing $f$, we get the Jacobian matrices needed by the EKF:
\begin{align}
    \mathbf{F}_t &=\begin{bmatrix}
        1 & 0 & -\Delta t v_{t} \sin(\hat{\theta}_{t \mid t}) \\
        0 & 1 & \Delta t v_{t} \cos(\hat{\theta}_{t \mid t}) \\
        0 & 0 & 1
    \end{bmatrix} \\
    \mathbf{H}_t &= \begin{bmatrix}
        1 & 0 & 0 \\
        0 & 1 & 0
    \end{bmatrix}
\end{align}
where $\hat{\theta}_{t \mid t}$ is the Kalman estimate of $\theta_t$ given measurements $\totmeas$. Not that here we are assuming that the measurements of the system are not distance measurements, but instead measurements of the robot positions. We get these positions measurements by using two optimization algorithms. 

% As described in section~\ref{sec:related-work}, this is a non-convex. However, the Riemannian Elevator~\cite{R_elevator} provides an approximate solution as well as a non-trivial lower bound on the problem which we can use to determine the quality of the solution. Given this, we can approximately solve the localization problem for the initial positions. The estimates can be further refined using gradient descent on the stress minimization problem as described in previous sections. 

Below in algorithm~\ref{algo:estimation}is a summary of the tracking pipeline. For conciseness, we define the function $\text{RE}(C, \tilde{D}, W)$ the output of the Riemannian elevator, and $\text{KA}(\hat{\mathbf{x}}_{t \mid t-1}, \mathbf{y}_t, \bm{\psi}_t)$ as the output of gradient descent using Kruskal's algorithm. 

\begin{algorithm}[ht]
    \caption{Online estimation}\label{algo:estimation}
    \textbf{Inputs:} Connectivity matrix $\mathbf{J}_0$, measurement vector $\mathbf{y}_0$, noise power vector $\bm{\psi}_0$
    
    \textbf{Initialize:} Get an initial guess
    \begingroup\setstretch{1.2} % Increase line spacing
    \begin{algorithmic}[1]
        \Statex \underline{Find initial guess with Riemannian elevator}
        \State Construct matrices $C$, $\tilde{D}$, and $W$, see \cite{R_elevator}
        \State $\hat{\mathbf{X}}_0 \ \leftarrow\ \text{RE}(C, \tilde{D}, W)$
        \State Recover $\hat{\mathbf{x}}_{0 \mid 0} \in \R{3n}$ from $\hat{\mathbf{X}}_0 \in \R{n \times 2}$, assuming $\hat{\theta}^{(i)} = 0$ with high uncertainty.
    \end{algorithmic}

    \textbf{Track:} Continuously track the states
    \begin{algorithmic}[1]
        \Statex \underline{Predict}
        \State $\hat{\mathbf{x}}_{t \mid t-1} = \mathbf{f}(\hat{\mathbf{x}}_{t-1 \mid t-1}, \mathbf{u}_{t-1})$
        \State $\mathbf{P}_{t \mid t-1} = \mathbf{F}_t \mathbf{P}_{t-1 \mid t-1} \mathbf{F}_t^\top + \mathbf{Q}_t$
        \Statex \underline{Measure}
        \State Collect distance measurements $\mathbf{y}_t$
        \State Get position measurements $\mathbf{z}_t = \text{KA}(\hat{\mathbf{x}}_{t \mid t-1}, \mathbf{y}_t, \bm{\psi}_t)$ and corresponding matrices and vectors $\mathbf{J}_t$, $\mathbf{L}_t$, $\bm{\psi}_t$
        \Statex \underline{Update}
        \State $\mathbf{K}_t = \mathbf{P}_{t \mid t-1} \mathbf{H}^\top_t (\mathbf{H}_t \mathbf{P}_{t \mid t-1} \mathbf{H}_t^\top + \mathbf{R}_t)^{-1}$
        \State $\hat{\mathbf{x}}_{t \mid t} = \hat{\mathbf{x}}_{t\mid t-1} + \mathbf{K}_t (\mathbf{z}_t - \mathbf{g}(\hat{\mathbf{x}}_{t \mid t-1}))$
        \State $\mathbf{P}_{t \mid t} = (\mathbf{I} - \mathbf{K}_t \mathbf{H}_t) \mathbf{P}_{t \mid t-1}$
        \Statex \underline{Repeat}: Go to \underline{Predict}
    \end{algorithmic}
    \endgroup
\end{algorithm}

\section{Results}

This section deals with the experimental results of the implemented algorithms. 

\subsection{Kruskal's Method}
An important part of the tracking pipeline is the gradient descent optimization step. To measure the performance of Kruskal's method, the speed and reliability was measured and compared with more established step schedules. In figures~\ref{fig:convergence-comp-small}, \ref{fig:convergence-comp-medium}, and~\ref{fig:convergence-comp-large}, these results are shown. 
\begin{figure}[ht]
    \centering
    \includesvg[width=\linewidth]{kruskal_comp_n4_m7.svg}
    \caption{Convergence speed of gradient descent for a small example. Kruskal's method is better. }
    \label{fig:convergence-comp-small}
\end{figure}
\begin{figure}[ht]
    \centering
    \includesvg[width=\linewidth]{kruskal_comp_n10_m45.svg}
    \caption{Convergence speed of gradient descent for a medium example. Performance is even. }
    \label{fig:convergence-comp-medium}
\end{figure}
\begin{figure}[ht]
    \centering
    \includesvg[width=\linewidth]{kruskal_comp_n15_m150.svg}
    \caption{Convergence speed of gradient descent for a large problem. Note that both $\frac{1}{\sqrt{k}}$ step and constant step size are missing the largest value because the objective value diverged.}
    \label{fig:convergence-comp-large}
\end{figure}
It is clear that Kruskal's method carries a performance advantage for smaller problems but it is outperformed by the other two methods for larger problems. However, it should be noted that, while Kruskal's method is superficially complex, it does not require much tuning to work well. The only parameter to choose is the initial step, whose value is not dependent on problem size. Meanwhile, for the other two algorithms to be faster, they require tuning for each specific problem instance. In the largest case shown in the plots, neither constant step size nor $\frac{1}{\sqrt{k}}$ step size converged for one the the chosen parameter values. This illustrates the perhaps most important property of Kruskal's algorithm: it works well for all problem sizes without tuning. 

\subsection{Initial Guess}
The Riemannian elevator was chosen to provide certifiable, accurate estimations of the initial positions, since the tracking performance hinges on the accuracy of this initial guess. In figure~\ref{fig:accuracy-comp}, the initial guess performance is shown. The plot was generated by running the Riemannian elevator to generate an initial guess, followed by running gradient descent as a refinement step. A trial was deemed a success if the final result matched the real arrangement of the robots, within some tolerance. For an example of a successful trial, see figure~\ref{fig:init-ex}
\begin{figure}[ht]
    \centering
    \includesvg[width=\linewidth]{accuracy_comparison.svg}
    \caption{Accuracy of the Riemannian elevator compared to a random (informed) guess.}
    \label{fig:accuracy-comp}
\end{figure}
\begin{figure}[ht]
    \centering
    \includesvg[width=\linewidth]{time_comparison.svg}
    \caption{Average runtime of the first estimation step in algorithm~\ref{algo:estimation} for a guess generated by the Riemannian elevator compared to a random guess.}
    \label{fig:speed-comp}
\end{figure}
As is shown in figure~\ref{fig:accuracy-comp}, the Riemannian elevator outperforms randomly guessing, which is to be expected. However, random initializations are still surprisingly accurate. Moreover, figure~\ref{fig:speed-comp} reveals the main issue with the algorithm, that being speed. It is at least an order of magnitude slower than gradient descent. 
\begin{figure}[ht]
    \centering
    \includegraphics[width=\linewidth]{point_est_85_10.png}
    \caption{This figure shows the initial guess generated by the Riemannian elevator and the estimates after refinement, as well as the measurements between the robots.}
    \label{fig:init-ex}
\end{figure}

\subsection{Tracking}
When using the Riemannian elevator and Kruskal's method together as in algorithm~\ref{algo:estimation}, it is possible to track the robots over time. In the plots below, the tracking performance of the pipeline is displayed. 
\begin{figure}[ht]
    \centering
    \includesvg[width=\linewidth]{m90_n10_s10.svg}
    \caption{Tracking performance for $n=10$ robots with $m=90$ measurements and low noise. The real state (blue line) is followed almost perfectly. The jumps in the orientation $\theta$ is due to modulo $2\pi$ plots. }
\end{figure}
\begin{figure}
    \centering
    \includesvg[width=\linewidth]{m120_n15_s50.svg}
    \caption{Tracking performance for $n=15$ robots with $m=120$ measurements and medium noise. The real state is in blue. The positions are tracked almost perfectly, while the angle is noisy. Note that the angle plot is modulo $2\pi$.}
\end{figure}

\section{Discussion}

\begin{itemize}
    \item Consistency
    \item Scheduled rerun of Riemannian elevator
    \item Computability of larger configurations
    \item Distributed? 
    \item Limitations, Future Work, and Conclusions
    \item Test with more realistic measurements
\end{itemize}


\section{Disclosure}

This is a joint project in the courses "AA273: State Estimation and Filtering for Robotic Perception" and "EE364b: Convex Optimization II".

\section{Codebase}

The codebase of this entire project, including this report, can be found on GitHub at \url{https://github.com/erikvah/spring-project}. 

\printbibliography

\end{document}

