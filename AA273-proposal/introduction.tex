In a variety of applications, it is necessary to localize a group of autonomous robots. In open-air environments, this can be done with GPS. In enclosed where GPS signal is not available, it is possible to do this using a variety of SLAM techniques such as bundle adjustment or pose graph optimization (PGO). These can and have been extended to optimize over a group of robots, simultaneously determining their positions~\cite{SLAM_distributed}. However, these are dependent on a complex signal processing pipeline to extract landmark or odometry measurements to optimize over. This project intends to investigate a different approach. 

The subject of this project is an implementation and extension of a novel optimization technique for distance-based localization. This problem is described in figure~\ref{fig:problem_desc}.
\begin{figure}[ht]
    \centering
    \includegraphics[width=\linewidth,trim=31mm 48mm 106mm 15mm, clip]{problem.pdf}
    \caption{Basic problem setup. In the figure, we have $N=4$ robots where $\bar{x}_i\in \mathbf{R}^n$ are the poses of the robots and $y_{i,j} \in \mathbf{R}$ are the distance measurements between them. }
    \label{fig:problem_desc}
\end{figure}
SLAM usually relies on measurements of external fixed points in the scene to extract measurements. This method would instead rely on the robots measuring only distance to each other. 
